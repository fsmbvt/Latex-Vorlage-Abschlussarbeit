% !TEX root = Thesis.tex

\chapter{Kapitel 1 }
\section{Section 1}
\blindtext
\begin{figure}[h!]
\begin{center}
\includegraphics[width=.5\textwidth]{example-image-duck} % takes a random page from the pdf
\caption{Testcaption}
\label{fig:ente}
\end{center}
\end{figure}

\begin{table}[h!]
\centering
\begin{tabular}{||c c c c||} 
 \hline
 Col1 & Col2 & Col2 & Col3 \\ [0.5ex] 
 \hline\hline
 1 & 6 & 87837 & 787 \\ 
 2 & 7 & 78 & 5415 \\
 3 & 545 & 778 & 7507 \\
 4 & 545 & 18744 & 7560 \\
 5 & 88 & 788 & 6344 \\ [1ex] 
 \hline
\end{tabular}
\caption{Table to test captions and labels.}
\label{table:1}
\end{table}

Dies ist eine Test-Zitation \cite{texbook} für das Literaturverzeichnis. Bilder und Tabellen werden folgendermaßen referenziert. Das Paket \textit{duckuments} gibt Testbilder mit einer Ente aus, wie in Abbildung \ref{fig:ente} dargestellt. Tabelle \ref{table:1} wird in der selben weise referenziert.