%%%%%%%%%%%%%
% Hast du das ReadMe.txt schon gelesen?
%%%%%%%%%%%%%

% ========= Standard Packages =========
\usepackage{xkeyval}    % für titelseite 
\usepackage[headsepline,plainheadsepline,draft=false]{scrlayer-scrpage} % Kopf- und Fußzeile Einstellungen
\usepackage{svg}

% ========= Dokumentinformationen =========
% für Metadaten ist hypersetup in Thesis.txt gedacht
\usepackage[
	hidelinks           %links nicht umrahmen 
]{hyperref}

% ========= Spracheinstellungen =========
\usepackage[utf8]{inputenc}         % encoding input Option 
\usepackage[T1]{fontenc}            % endcoding output Option 
\usepackage[ngerman]{babel}         % gewählte Dokumentsprache(n)
\usepackage{lmodern}                % Schriftart (Latin Modern)

% ========= Datumseinstellungen =========
\usepackage[ngerman]{datetime}

% ========= Grafiken Pakete =========
\usepackage{graphicx}               % einbinden von Grafiken bzw. Bildern
\usepackage{subcaption}             % caption für mehrere subfigures

% ========= Tabellen Pakete =========
\usepackage{longtable}              % Mehrseitige Tabellen
\usepackage{tabularx}               % flexiblere Tabellenumgebung
\usepackage{multirow}               
\usepackage{multicol}

% ========= sonstige Pakete =========
\usepackage[justification=centering]{caption}       % erweiterte Mögichkeiten für captions
\usepackage{abstract}                               % Umgebung für die Kurzfassung /Abstract
\usepackage[printonlyused,nohyperlinks]{acronym}    % Abkürzungsverzeichnis 
\usepackage{xcolor}                                 % Farboptionen
\usepackage[section]{placeins}                      % ermöglicht den Befehls \FloatBarrier
\usepackage{csquotes}                               % passenden Anführungszeichen für Zitate

% ========= Bibliography =========
\usepackage[backend=biber,
            style=ieee,
            dashed=false,
            maxcitenames=2,
            mincitenames=1]{biblatex}

\addbibresource{Literaturverzeichnis.bib}       % Pfad zur .bib Datei

% ========= DEBUG Packages ========
\usepackage{blindtext}
\usepackage{duckuments}
%\usepackage{showframe}% blendet Seitenränder ein


% !!! Do not include packages beyond this line !!!
%%%%%%%%%%%%%%%%%%%%%%%%%%%%%%%%%%%%%%%%%%%%%%%%%%%%%%%%%%%%%%%%%%%%%%%%%%%%%%
% ========= benutzerdefinierte commands =========

%nicht einrücken nach Absatz
\setlength{\parindent}{0pt}
\RedeclareSectionCommand[afterindent=false,beforeskip=10pt]{chapter}
 % neues Datumformat
\newdateformat{mydateformat}{\monthnamengerman[\THEMONTH] \THEYEAR}

% Dokument Kopf- und Fußzeile
\clearpairofpagestyles
\pagestyle{scrheadings}
\ihead*{\headmark}
\automark{chapter}
\ohead*{\pagemark}